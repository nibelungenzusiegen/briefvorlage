\documentclass[%
  fontsize=12pt, % Schriftgröße
  version=last%  % Neueste Version von KOMA-Skript verwenden
]{scrlttr2}

% ===== Deutsche Sprache =====
\usepackage[utf8]{inputenc}
\usepackage[ngerman]{babel}
% ============================

\LoadLetterOption{DIN} % Einstellungen für DIN 676 laden

\LoadLetterOption{absender} % Absenderdaten und -einstellungen aus absender.lco laden

\usepackage{graphicx} % Um Grafiken (bspw. das Logo) einbinden zu können

\begin{document}

\begin{letter}{%
% ===== Zielanschrift =====
  Erika Musterfrau\\
  Musterweg 43\\
  56789 Musterhausen%
% =======================
}

% ====== Geschäftszeichenzeile =========
\setkomavar{yourref}{}          % Ihr Zeichen
\setkomavar{yourmail}{}         % Ihr Schreiben vom
\setkomavar{myref}{}            % Unser Zeichen
\setkomavar{customer}{}         % Kundennummer
\setkomavar{invoice}{}          % Rechnungsnummer
\setkomavar{place}{Musterstadt} % Ort
\setkomavar{date}{\today}       % Datum
% =====================================

\setkomavar{title}{Titel}

\setkomavar{subject}{Betreff}

\opening{Sehr geehrte Frau Musterfrau,}

hier kommt der Text hin.

\closing{Mit freundlichen Grüßen,}

% ===== Postskriptum =====
\ps PS: \dots
% ========================

% ===== Anlage(n) =====
% \setkomavar*{enclseparator}{Anlage}
\encl{%
  Anlage 1\\
  Anlage 2%
}
% ===================

% ===== Verteiler =====
% \setkomavar*{ccseparator}{Kopie an}
\cc{%
  Verteiler 1\\
  Verteiler 2%
}
% =====================

\end{letter}
\end{document}
