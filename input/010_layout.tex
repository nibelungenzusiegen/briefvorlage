\LoadLetterOption{DIN} % Einstellungen für DIN 676 laden

\newcommand \vulgo {$^{v}$/~}

%\LoadLetterOption{absender} % Absenderdaten und -einstellungen aus absender.lco laden
%\ProvidesFile{absender.lco}

\KOMAoptions{%
% fromemail=true,       % Email wird im Briefkopf angezeigt
% fromphone=true,       % Telefonnumer wird im Briefkopf angezeigt
% fromfax=false,         % Faxnummer wird im Briefkopf angezeit
% fromurl=true,         % URL wird im Briefkopf angezeigt
fromlogo=true,        % Logo wird im Briefkopf angezeigt
% subject=titled,       % Druckt "Betrifft: " vor dem Betreff
locfield=wide,          % Breite Absenderergänzung (location)
fromalign=left,         % Ausrichtung des Briefkopfes
fromrule=afteraddress  % Trennlinie unter dem Briefkopf
}

% ===== Absenderergänzung =====
\setkomavar{location}{%
  \raggedright\footnotesize{%
  \usekomavar{fromname}\\
  \usekomavar{fromaddress}\\
  \usekomavar*{fromphone}\usekomavar{fromphone}\\
%  \usekomavar*{fromfax}\usekomavar{fromfax}\\
  \usekomavar*{fromemail}\usekomavar{fromemail}
  \usekomavar*{fromurl}\usekomavar{fromurl}}%
}
% ============================
   

% Logo
\setkomavar{fromlogo}{\includegraphics[width=2cm]{images/wappen.png}}

\setkomavar{firsthead}[c]{
\noindent
\parbox[][][t]{2cm}{\usekomavar{fromlogo}} 
\parbox[][][s]{1\textwidth}{
	{\large \gprefix}		\\
	{\LARGE \gname}			\\
	{\large \gsuffix}		} 
}

% Die Bankverbindung wird nicht automatisch verwendet. Dazu muss bspw. mittels \firstfoot ein eigener Brieffuß definiert werden.
\setkomavar{frombank}{}

% ===== Signatur =====
\setkomavar{signature}{%
  \gsenderamt\\
  \gsender ~\vulgo \gsenderbierspitz  ~ \parbox{0.5cm}{\includegraphics[width=0.5cm]{images/zirkel.eps}}
}
\renewcommand*{\raggedsignature}{\raggedright}
% ====================